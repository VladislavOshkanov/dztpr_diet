%эти строчки трогать без знания дела не стоит.
\documentclass[]{article}
\usepackage[utf8]{inputenc}
\usepackage[russian]{babel}
\usepackage{amsmath}
\usepackage{graphicx}
\hoffset=-100pt
\voffset=-100pt
\textwidth=520pt
\textheight=700pt
%а тут уже можно что-нибудь менять.
\title{Задача о рационе}
\author{Ошканов В.С. гр. 13122}
\begin{document}
\maketitle
\section{Математическая модель}

Список адресов:

1. Мусы Джалиля 11

2. просп. Академика Коптюга 13

3. Академическая 34

4. Пирогова 2

5. Пирогова 16

6 Просп. Академика Лаврентьева 6
\begin{table}\caption{Расстояние между пунктами}\label{tab:1}
\centering
\begin{tabular}{|c|c|c|c|c|c|c|}
	\hline
- &    1  &   2  &   3  &   4  &   5  &  6  \\
\hline
1 &   -  & 3300 & 4100 & 3100 & 2300 & 2000 \\
\hline
2 & 3300 &  -   & 2500 &  690 & 1400 & 1400 \\
\hline
3 & 4100 & 2500 &  -   & 2500 & 3100 & 2200 \\
\hline
4	& 3100 &  690 & 2500 &  -   & 760  & 1800 \\
\hline
5	& 2300 & 1400 & 3100 &  760 &  -   & 2100 \\
\hline
6 & 2000 & 1400 & 2200 & 1800 & 2100 &  -   \\
\hline

\hline
\end{tabular}
\end{table}



Я решаю задачу коммивояжера.
Для записи математической модели нам потребуются следующие обозначения:
\par
\textit{Множества}:
\par\noindent
$F$ --- множество продуктов.
\par\noindent
$N$ --- множество питательных веществ.
\par
\textit{Параметры}:
\par\noindent
$p_i$ --- цена у продукта $i$, $i\in F$;
\par\noindent
$d$ --- количество дней, на которые необходимо купить продукты.
\par\noindent
$b_{j}$ --- минимальная рекомендуемая суточная норма потребления питательного вещества $j$, $j\in N$.
\par\noindent
$a_{ij}$ --- содержание питательного вещества $j$ в продукте $i$, $i\in F$, $j\in N$.
\par
\textit{Переменные}:
\par\noindent
$q_i$ --- количество продуктов, которое необходимо покупать каждые $d$ дней в
единицах продукта (кг, шт, л, десяток).
С использованием введенных обозначений, математическая модель задачи о рационе
запишется следующим образом:
\begin{equation}
\sum_{i\in F}p_iq_i\rightarrow\min;
\end{equation}
\begin{equation}\label{eq:2}
	\sum_{i\in F}
			q_ia_{ij} \geq b_j*d,\quad j\in N,
\end{equation}
\begin{equation}
    q_i \geq 0,\quad i\in F.
\end{equation}
С помощью неравенства \eqref{eq:2} я обозначил, что количество питательных
веществ в покупаемых продуктах должно превосходить минимальную суточную
потребность.

\section{Вычислительный эксперимент}
Я взял следующие продукты, содержащие следующие питательные вещества:
у меня получилась таблица \ref{tab:1}.
\begin{table}\caption{содержание полезных веществ}\label{tab:1}
\centering
\begin{tabular}{|c|c|c|c|c|c|c|c|c|c|}
\hline
Название & Ед. изм & Белки(г) & Жиры(г) & Углеводы(г) & Железо(мг) & Кальций(мг) & Цена(р) & Ограничение\\
\hline
Говядина & 100г & 26 & 15 & 0 & 2.6 & 18 & 28.6 & 400г\\
\hline
Хлеб & булка(700г) & 63 & 22.7& 343 & 25.2 & 1820 & 26 & 1 булка\\
\hline
Яйцо & десяток & 65 & 55 & 5.5 & 6 & 250 & 52 & 0.2 дес\\
\hline
Огурец & 100г & 0.7 & 0.1 & 3.6 & 0.3 & 16.0 & 10.5 & 400г \\
\hline
Яблоко & шт & 0.6 & 0.4 & 28 & 0.2 & 12 & 16.6 & 2шт\\
\hline
Масло & уп(180г) & 1.8 & 145.8 & 0.18 & 0 & 43.2 & 110 & 0.2уп\\
\hline
Творог & уп(220г) & 39.6 & 19.8 & 6.6 & 0.88 & 360.8 & 104 & 1уп\\
\hline
Картофель & 100г & 2.0 & 0.1 & 17.0 & 0.8 & 12.0 & 1.7 & 400г\\
\hline
Морковь & 100г & 0.9 & 0.2 & 10 & 0.3 & 33 & 1.9 & 400г\\
\hline
Рис & уп(500г) & 13.5 & 1.5 & 280 & 1 & 50 &86 & 0.5 уп\\
\hline
Молоко & л & 34 & 1 & 5 & 0 &125 & 67 & 0.8л\\
\hline
\end{tabular}
\end{table}


Рассчитав необходимое количество продуктов для жизни в течение 30 дней я получил результат, обозначенный в таблице \ref{tab:2}.
\begin{table}
	\caption{Первый результат}
	\centering
	\label{tab:2}
	\begin{tabular}{|c|c|}
		\hline
		Название&Количество\\
		\hline
		Хлеб & 13 булок\\
		\hline
		Масло & 12 упаковок(180г)\\
		\hline
		Картофель & 5.6кг\\
		\hline
	\end{tabular}

\end{table}
За продукты нужно отдать 1753 рубля. Результат интересный, но неприменимый для жизни.  В выборку вошли продукты с
максимальным отношением количества питательных веществ к цене. Невозможно питаться всего тремя продуктами,
причем количество углеводов было превышено почти в 2 раза. Далее, я решил обозначить не только минимальную,
но и максимальную суточную потребность. Данные я так же взял из википедии. К модели добавился один новый параметр и
новые ограничения:

\par\noindent
$m_j$ --- максимальная рекомендуемая суточная потребность в питательном веществе $j$,$\quad j\in N$.
\begin{equation}
	\sum_{i\in F}q_ia_{ij} \leq m_j*d,\quad j\in N.
\end{equation}

Результат получился более разнообразным, а цена сразу выросла до 2859 рублей. В
выборку вошли на этот раз 6 продуктов и диета стала более "здоровой". Результат
в таблице \ref{tab:3}.
Получилось, что в месяц нужно съедать 28 кг морковки, это почти килограмм в день,
вряд ли это кому-то понравится. Поэтому я решил добавить ограничение на количество
каждого продукта исходя из здравого смысла. Ограничения можно посмотреть в таблице~\ref{tab:1}.
Добавился ещё один параметр и следующие ограничения:
\par\noindent
$k_i$ --- максимальное потребление продукта $i$ в день (в единицах продукта),$\quad i\in F$.
\begin{equation}
	q_i \leq k_i*d,\quad i\in F.
\end{equation}
\begin{table}
	\caption{Второй результат}
	\centering
	\label{tab:3}
	\begin{tabular}{|c|c|}
		\hline
		Название&Количество\\
		\hline
		Хлеб & 6 булок\\
		\hline
		Масло & 13 упаковок(180г)\\
		\hline
		Картофель & 1.3кг\\
		\hline
		Морковь & 28кг\\
		\hline
		Огурец & 4.7кг\\
		\hline
		Яблоко & 12шт\\
		\hline

	\end{tabular}

\end{table}

Ну вот, с этим уже можно жить, правда и цена соответствующая, 6296 рублей в месяц... Результат в таблице~\ref{tab:4}. Лидирует, по прежнему,
морковка, но в рацион уже вошли почти все продукты.
\begin{table}
	\caption{Финальный результат}
	\centering
	\label{tab:4}
	\begin{tabular}{|c|c|}
		\hline
		Название&Количество\\
		\hline
		Говядина & 1.3 кг\\
		\hline
		Хлеб & 8 булок\\
		\hline
		Яйцо & 6 штук\\
		\hline
		Огурец & 600г\\
		\hline
		Масло & 6 упаковок(180г)\\
		\hline
		Картофель & 100г\\
		\hline
		Морковь & 11.2кг\\
		\hline
		Яблоко & 59шт\\
		\hline
		Рис & 13упаковок(500г) \\
		\hline
		Творог & 23 упаковки (220г)\\
		\hline

	\end{tabular}

\end{table}
\section{Вывод}
При самых простых и очевидных ограничениях результат совсем не подходит для
реального человека. Но, анализируя результаты и накладывая дополнительные
ограничения, можно получить уже что-то более премлемое. Но для того, чтобы
применять результаты в жизни, нужно рассматривать намного больше продуктов,
питательных веществ, витаминов и микроэлементов, учитывать усваиваемость
каждого вещества в каждом продукте. Для этого необходимо много времени и
серьёзные исследования.
\end{document}
