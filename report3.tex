\documentclass[]{article}
\usepackage[utf8]{inputenc}
\usepackage[russian]{babel}
\usepackage{amsmath}
\usepackage{graphicx}
\hoffset=-100pt
\voffset=-100pt
\textwidth=520pt
\textheight=700pt
%а тут уже можно что-нибудь менять.
\title{The bus-driver-planning challenge}
\author{Ошканов В.С. гр. 13122}
\begin{document}
\maketitle
\section{Математическая модель}

Я решаю задачу о составлении расписания водителей автобусов с
сайта https://careers.quintiq.com/puzzle.html
Целью задачи является составление расписания с максимальной
суммой бонусов и штрафов, обозначенных в задании.

Для записи математической модели нам потребуются следующие обозначения:
\par
\textit{Множества}:
\par\noindent
$I$ --- множество водителей.
\par\noindent
$J$ --- множество дней.
\par\noindent
$K$ --- множество дней.
\par
\textit{Параметры}:
\par\noindent
$d_{ij}$ --- (d - day off) равна 1, если у водителя $i$ выходной в день $j$, и 0
в противном случае, $i\in I$, $j\in J$.
\par\noindent
$pd_{ij}$ --- (pd - prefer day off) равен 1, если водитель $i$ хочет отдыхать в
день $j$, $i\in I$, $j\in J$.
\par\noindent
$pe_{ij}$ --- (pe - prefer early) водитель $i$ хочет работать в утреннюю
смену в день $j$, $i\in I$, $j\in J$.
\par\noindent
$pl_{ij}$ --- (pl - prefer late) водитель $i$ хочет работать в вечернюю
смену в день $j$, $i\in I$, $j\in J$.
\par\noindent
$can_{ik}$ --- квалификация водителя $i$ позволяет ему работать на маршруте $k$,
 $i\in I$, $k\in K$.
\par
\textit{Переменные}:
\par\noindent
$e_{ijk}$ --- равна 1, если водитель $i$ работает в день $j$ на маршруте $k$
в утреннюю смену.
\par\noindent
$l_{ijk}$ --- равна 1, если водитель $i$ работает в день $j$ на маршруте $k$
в вечернюю смену.

Далее следует набор бинарных переменных, необходимых для составления модели:
$earlyafterlate_{ij}$ - равна 1, если у водителя утренняя смена следует за вечерней,
$i \in I, j in J\\{14}$.
\par\noindent
$earlyshiftspare_{jk}$ - равна 1, если в день $j$ на маршруте $k$ нет водителя в утреннюю смену.
\par\noindent
$lateshiftspare_{jk}$ - равна 1, если в день $j$ на маршруте $k$ нет водителя в вечернюю смену.
\par\noindent
$morethanfourdays_{i}$ - равна тому, насколько больше, чем 4,  вечерних смен водитель $i$ выполняет за все дни.
\par\noindent
$lessthanfourdays_{i}$ - равна тому, насколько меньше, чем 4,  вечерних смен водитель $i$ выполняет за все дни.
\par\noindent
$threedaysoff_{i in I, j in J}$ - водитель $i$ имеет 3 выходных подряд, начиная с дня $j$.
\par\noindent

С использованием введенных обозначений, математическая модель задачи о рационе
запишется следующим образом:
\begin{eqnarray}
-4 * \sum_{i\in I}\sum_{j\in J}(pd_{ij}*\sum_{k \in K}(e_{ijk}+l_{ijk}))
-8 * \sum_{i\in I}(morethanfourdays_i+lessthanfourdays_i) \\ \nonumber
-30 * \sum_{i\in I}\sum_{j in J\backslash{14}} earlyafterlate_ij
-20 * \sum_{j \in J}\sum_{k \in K} (earlyshiftspare_{ik} + lateshiftspare_{jk}) \\ \nonumber
+ 3 * \sum_{i \in I}\sum_{j \in J}\sum_{k \in K} (pe_{ij}*e_{ijk} + pl_{ij}*l_{ijk})
+ 5 * \sum_{i \in I}\sum_{j \in J\backslash{13,14}} threedaysoff_{ij}\rightarrow\max;
\end{eqnarray}
\begin{equation}
  \sum_{i \in I} e_{ijk} + earlyshiftspare_{jk} = 1, j \in J, k \in K.
\end{equation}
\begin{equation}
  \sum_{i \in I} l_{ijk} + lateshiftspare_{jk} = 1, j \in J, k \in K.
\end{equation}
\begin{equation}
  \sum_{k \in K} e_{ijk} \leq 1, i \in I, j \in J.
\end{equation}
\begin{equation}
  \sum_{k \in K} l_{ijk} \leq 1, i \in I, j \in J.
\end{equation}
\begin{equation}
  \sum_{k \in K} e_{ijk}*d_{ij} = 0, i \in I, j \in J.
\end{equation}
\begin{equation}
  \sum_{k \in K} l_{ijk}*d_{ij} = 0, i \in I, j \in J.
\end{equation}
\begin{equation}
  \sum_{n = j}^{j+2}\sum_{k \in K} l_{ink} \leq 3, i \in I, j \in J\backslash\{13,14\}.
\end{equation}
\begin{equation}
  \sum_{k \in K} (e_{i j+1 k} + l_{ijk})
              - earlyafterlate_{ij} \leq 1, i \in I, j in J\backslash\{13\}.
\end{equation}
\begin{equation}
  \sum_{k \in K} (e_{ijk}+l_{ijk}) \leq 1, i \in I, j \in J.
\end{equation}
\begin{equation}
  \sum_{j \in J}\sum_{k \in K} l_{ijk} - morethanfourdays_{i} - 4 \leq 0, i \in I.
\end{equation}
\begin{equation}
  \sum_{j \in J}\sum_{k \in K} l_{ijk} - lessthanfourdays_{i} - 4 \geq 0, i \in I.
\end{equation}
\begin{equation}
  e_{ijk}*(1-can_{ik}) = 0, i \in I, j \in J, k \in K.
\end{equation}
\begin{equation}
  l_{ijk}*(1-can_{ik}) = 0, i \in I, j \in J, k \in K.
\end{equation}
\begin{equation}
  \sum_{k \in K} (e_{ijk}+l_{ijk}+e{ij+1k}+l_{ij+1k}+e_{ij+2k}+l_{ij+2k})
        + 3*threedaysoff_{ij} \leq 3, i \in I, j \in J\backslash\{13,14\}.
\end{equation}
\begin{equation}
  \sum {j \in  J\backslash\{13,14\}} threedaysoff_{ij} \leq 1.
\end{equation}
С помощью неравенства \eqref{eq:2} я обозначил, что количество питательных
веществ в покупаемых продуктах должно превосходить минимальную суточную
потребность.

\section{Вычислительный эксперимент}
Я взял следующие продукты, содержащие следующие питательные вещества:
у меня получилась таблица \ref{tab:1}.



Рассчитав необходимое количество продуктов для жизни в течение 30 дней я получил результат, обозначенный в таблице \ref{tab:2}.


За продукты нужно отдать 1753 рубля. Результат интересный, но неприменимый для жизни.  В выборку вошли продукты с
максимальным отношением количества питательных веществ к цене. Невозможно питаться всего тремя продуктами,
причем количество углеводов было превышено почти в 2 раза. Далее, я решил обозначить не только минимальную,
но и максимальную суточную потребность. Данные я так же взял из википедии. К модели добавился один новый параметр и
новые ограничения:

\par\noindent
$m_j$ --- максимальная рекомендуемая суточная потребность в питательном веществе $j$,$\quad j\in N$.
\begin{equation}
	\sum_{i\in F}q_ia_{ij} \leq m_j*d,\quad j\in N.
\end{equation}

Результат получился более разнообразным, а цена сразу выросла до 2859 рублей. В
выборку вошли на этот раз 6 продуктов и диета стала более "здоровой". Результат
в таблице \ref{tab:3}.
Получилось, что в месяц нужно съедать 28 кг морковки, это почти килограмм в день,
вряд ли это кому-то понравится. Поэтому я решил добавить ограничение на количество
каждого продукта исходя из здравого смысла. Ограничения можно посмотреть в таблице~\ref{tab:1}.
Добавился ещё один параметр и следующие ограничения:
\par\noindent
$k_i$ --- максимальное потребление продукта $i$ в день (в единицах продукта),$\quad i\in F$.
\begin{equation}
	q_i \leq k_i*d,\quad i\in F.
\end{equation}
\begin{table}
	\caption{Второй результат}
	\centering
	\label{tab:3}
	\begin{tabular}{|c|c|}
		\hline
		Название&Количество\\
		\hline
		Хлеб & 6 булок\\
		\hline
		Масло & 13 упаковок(180г)\\
		\hline
		Картофель & 1.3кг\\
		\hline
		Морковь & 28кг\\
		\hline
		Огурец & 4.7кг\\
		\hline
		Яблоко & 12шт\\
		\hline

	\end{tabular}

\end{table}

Ну вот, с этим уже можно жить, правда и цена соответствующая, 6296 рублей в месяц... Результат в таблице~\ref{tab:4}. Лидирует, по прежнему,
морковка, но в рацион уже вошли почти все продукты.
\begin{table}
	\caption{Финальный результат}
	\centering
	\label{tab:4}
	\begin{tabular}{|c|c|}
		\hline
		Название&Количество\\
		\hline
		Говядина & 1.3 кг\\
		\hline
		Хлеб & 8 булок\\
		\hline
		Яйцо & 6 штук\\
		\hline
		Огурец & 600г\\
		\hline
		Масло & 6 упаковок(180г)\\
		\hline
		Картофель & 100г\\
		\hline
		Морковь & 11.2кг\\
		\hline
		Яблоко & 59шт\\
		\hline
		Рис & 13упаковок(500г) \\
		\hline
		Творог & 23 упаковки (220г)\\
		\hline

	\end{tabular}

\end{table}
\section{Вывод}
При самых простых и очевидных ограничениях результат совсем не подходит для
реального человека. Но, анализируя результаты и накладывая дополнительные
ограничения, можно получить уже что-то более премлемое. Но для того, чтобы
применять результаты в жизни, нужно рассматривать намного больше продуктов,
питательных веществ, витаминов и микроэлементов, учитывать усваиваемость
каждого вещества в каждом продукте. Для этого необходимо много времени и
серьёзные исследования.
\end{document}
