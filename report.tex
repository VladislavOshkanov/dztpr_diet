%эти строчки трогать без знания дела не стоит.
\documentclass[]{article}
\usepackage[utf8]{inputenc}
\usepackage[russian]{babel}
\usepackage{amsmath}
\usepackage{graphicx}
\textwidth=520pt
\textheight=700pt
%а тут уже можно что-нибудь менять.
\title{Задача о рационе}
\author{Ошканов В.С. гр. 13122}
\begin{document}
\maketitle
\section{Математическая модель}
Я решаю задачу о рационе. Цены на продукты я взял на сайте extremeshop.ru, а рекомендуемую суточную норму потребления из википедии.

Для записи математической модели нам потребуются следующие обозначения.
\par
\textit{Множества}:
\par\noindent
$F$ --- множество продуктов.
\par\noindent
$N$ --- множество питательных веществ.
\par
\textit{Параметры}:
\par\noindent
$p_i$ --- цена у продукта $i$, $i\in F$;
\par\noindent
$d$ --- количество дней, на которые необходимо купить продукты.
\par\noindent
$b_{j}$ --- рекомендуемая суточная норма потребления питательного вещества $j$, $j\in N$.
\par\noindent
$a_{ij}$ --- содержание питательного вещества $j$ в продукте $i$, $i\in F$, $j\in N$.
\par
\textit{Переменные}:
\par\noindent
$q_i$ --- количество продуктов, которое необходимо покупать каждые $d$ дней
единицах продукта (кг, шт, л, десяток ...).
С использованием введенных обозначений, математическая модель задачи о рационе
запишется следующим образом:
\begin{equation}
\sum_{i\in F}p_iq_i\rightarrow\min;
\end{equation}
\begin{equation}\label{eq:2}
	\sum_{i\in F}
			q_ia_{ij} \geq b_i*d,\quad j\in N,
\end{equation}
\begin{equation}
      q_i \geq 0,\quad i\in F.
\end{equation}
С помошью неравенства \eqref{eq:2} я обозначил, что количество питательных веществ в покупаемых продуктах должно превосходить суточную норму.

\section{Вычислительный эксперимент}
Я взял следующие продукты, содержащие следующие питательные вещества:
у меня получилась таблица \ref{tab:1}.
\begin{table}\caption{содержание полезных веществ}\label{tab:1}
\centering
\begin{tabular}{|c|c|c|c|c|c|c|c|c|}
\hline
Название & Ед. изм & Белки(г) & Жиры(г) & Углеводы(г) & Железо(мг) & Кальций(мг) & Цена(р)\\
\hline
Говядина & 100г & 26 & 15 & 0 & 2.6 & 18 & 28.6\\
\hline
Хлеб & булка & 63 & 22.7& 343 & 25.2 & 1820 & 26\\
\hline
Яйцо & десяток & 65 & 55 & 5.5 & 6 & 250 & 52\\
\hline
Огурец & 100г & 0.7 & 0.1 & 3.6 & 0.3 & 16.0 & 10.5\\
\hline
Яблоко & шт & 0.6 & 0.4 & 28 & 0.2 & 12 & 16.6\\
\hline
Масло & уп & 1.8 & 145.8 & 0.18 & 0 & 43.2 & 110\\
\hline
Творог & уп & 39.6 & 19.8 & 6.6 & 0.88 & 360.8 & 104\\
\hline
Картофель & 100г & 2.0 & 0.1 & 17.0 & 0.8 & 12.0 & 1.7\\
\hline
Морковь & 100г & 0.9 & 0.2 & 10 & 0.3 & 33 & 19\\
\hline
Рис & уп(500г) & 13.5 & 1.5 & 280 & 1 & 50 &86\\
\hline
Молоко & л & 34 & 1 & 5 & 0 &125 & 67\\
\hline
\end{tabular}
\end{table}

%Если бы я захотел, я бы вставил сюда картинку:
%\begin{figure}
%\includegraphics{pic.jpg}
%\end{figure}
Рассчитав необходимое количество продуктов для жизни в течение 30 дней я получил результат, обозначенный в таблице \ref{tab:2}.
\begin{table}
	\caption{Первый результат}
	\centering
	\label{tab:2}
	\begin{tabular}{|c|c|}
		\hline
		Название&Количество\\
		\hline
		Хлеб & 13 булок\\
		\hline
		Масло & 12 упаковок(180г)\\
		\hline
		Картофель & 5.6кг\\
		\hline
	\end{tabular}

\end{table}
За продукты нужно отдать 1753 рубля. Результат интересный, но неприменимый для жизни.  В выборку вошли продукты, с
максимальным отношением питательного вещества к цене. Невозможно питаться всего тремя продуктами,
причем количество углеводов было превышено почти в 2 раза. Далее, я решил обозначить не только минимальную,
но и максимальную суточную потребность. Данные я так же взял из википедии. К модели добавился один новый параметр и одно
новое ограничение.
\par\noindent
$m_j$ --- максимальная рекомендуемая суточная потребность в питательном веществе j.
\begin{equation}
	q_ia_{ij} \leq m_i*d,\quad j\in N,
\end{equation}
Это было невероятно, мне очень понравилось, буду всегда решать задачи дискретной оптимизации!

\end{document}
